\chapter{Allgemeine Erg"anzungen}

\section{ChatGPT}
\label{chatgpt}
Im folgenden ist eine Interaktion mit GPT-3.5 protokolliert. Auf die \gls{ki} wurde �ber die Webseite \cite{openaiChatGPT} zugegriffen. Das Vorgehen und die Leitfragen aus dem Prompt sind aus \cite{tiffertGrundlagenOperativenVertriebsmanagement2020} entnommen.
\subsubsection{Prompt}
Ich habe folgenden Prozess zur Teilnahme an einer Ausschreibung f�r einen Auftrag eines Ministeriums:\\
Der Prozess zur Neukundengewinnung beziehungsweise der Teilnahme an einer Ausschreibung k�nnte mit der Suche nach geeigneten Ausschreibungen beginnen. Darauf w�rde eine Evaluierung der Ausschreibungen auf Machbarkeit folgen. Danach k�men gegebenenfalls Angebotserstellung und -abgabe und im Falle einer Zusage eine Vertragserstellung und -unterzeichung.

Bitte definiere mir Kompetenzen, die zur Umsetzung der konkreten Aufgaben in dem Prozess notwendig sind. Nutze daf�r folgendes Vorgehen:\\
1. Reflexion der Schl�sselaufgaben im Rahmen der Verkaufsprozesse, 2. Reflexion der erfolgskritischen Verhaltensweisen, 3. Ableitung von Kompetenzen, 4. Formulierung von definierenden Verhaltensbeschreibungen.

Nutze f�r 2. die Leitfrage: Was muss man machen, um die Schl�sselaufgaben erfolgreich zu bew�ltigen?\\
Nutze f�r 3. die Leitfragen: Welche der obigen Beschreibungen zielen im Kern auf eine gleiche Grundverhaltensweise? Wie kann die jeweilige Gruppe mit einem zentralen Begriff bezeichnet werden?
\subsubsection{Antwort}
Hier ist eine m�gliche Herangehensweise:

Reflexion der Schl�sselaufgaben im Rahmen der Verkaufsprozesse:\\
a. Suche nach geeigneten Ausschreibungen\\
b. Evaluierung der Ausschreibungen auf Machbarkeit\\
c. Angebotserstellung und -abgabe\\
d. Vertragserstellung und -unterzeichnung\\

Reflexion der erfolgskritischen Verhaltensweisen:

Was muss man machen, um die Schl�sselaufgaben erfolgreich zu bew�ltigen?\\
a. Effektive Recherche- und Analysef�higkeiten, um geeignete Ausschreibungen zu identifizieren und zu verstehen.\\
b. Bewertungskompetenz, um die Machbarkeit der Ausschreibungen zu beurteilen und Potenziale sowie Risiken zu erkennen.\\
c. �berzeugende Kommunikationsf�higkeiten f�r die Erstellung und Pr�sentation von Angeboten.\\
d. Verhandlungsgeschick und rechtliches Verst�ndnis f�r die Vertragsverhandlungen und -abschl�sse.\\
Ableitung von Kompetenzen:

Welche der obigen Beschreibungen zielen im Kern auf eine gleiche Grundverhaltensweise?\\
a. Recherche- und Analysef�higkeiten sowie Bewertungskompetenz k�nnen als \glqq{}Informationsverarbeitung\grqq{} zusammengefasst werden.\\
b. Kommunikationsf�higkeiten und Verhandlungsgeschick k�nnten als \glqq{}Interaktion und �berzeugung\grqq{} betrachtet werden.\\
Formulierung von definierenden Verhaltensbeschreibungen:\\
a. Informationsverarbeitung: F�higkeit zur effizienten Recherche und Analyse von Ausschreibungen, um relevante Informationen zu extrahieren und zu bewerten.\\
b. Interaktion und �berzeugung: F�higkeit zur klaren und �berzeugenden Kommunikation sowie zum erfolgreichen Verhandeln w�hrend des Angebots- und Vertragsprozesses.\\

Diese beiden Gruppen von Kompetenzen - Informationsverarbeitung und Interaktion/�berzeugung - sind entscheidend f�r den erfolgreichen Abschluss des Ausschreibungsprozesses und k�nnen als grundlegende Anforderungen f�r die Teilnahme an Ausschreibungen betrachtet werden.
