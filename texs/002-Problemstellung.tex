\chapter{Problemstellung}
\plannedpages{3 - 6 | T: 16.11.25}

Die \gls{deg}\footnote{Bei der Kleinschreibung des Namens handelt es sich um ein Stilmittel \citep[vgl.][]{dataexpertsgmbhImpressumDataExperts}.} produziert einen Webclient, mit dem Landwirte F�rdermittel beantragen k�nnen.
In diesem f�llen Anwender Antragsformulare aus und hinterlegen Fl�chendaten im \glspl{gis}, um F�rdermittel zu beantragen. 

Der Webclient erleichtert das Ausf�llen der Formulare durch eine Live-Validierung der Antragsdaten. Er gibt eine Meldung aus, wenn die Datenkonstellation bestimmte Bedingungen erf�llt. Die Meldungen enthalten Bearbeitungshinweise oder Fehlermeldungen, die Antragsteller dabei unterst�tzen, inhaltlich korrekte Antr�ge einzureichen.

Die Live-Validierungen werden von den Autraggebern definiert. Daf�r verwenden sie den sogenannten Formularverwalter. In diesem werden neben den Live-Validierungen Vorgaben wie Einreichzeitr�ume, Formularlayouts und Vortragealgorithmen erfasst.

\section{Plausibilit�tspr�fungen}
F�r eine Live-Validierung (auch Plausibilit�tspr�fung genannt) werden im Formularverwalter folgende Eigenschaften definiert: Name, Fehlertext, Beschreibung, Fehlerlevel und verkn�pfte Formulare und Antragsverfahren.

\subsubsection{Name}
Der Name ist der Bezeichner der Vorgabe. Er ist nicht eindeutig -- es kann mehrere Vorgaben mit dem gleichen Namen geben. Die Auftraggeber sind dazu aufgefordert die Namen technisch auswertbar zu gestalten, indem sie auf die Verwendung von Umlauten, Sonderzeichen und Leerzeichen im Namen verzichten. Das wird nicht erzwungen und deshalb nicht bei allen Vorgaben ber�cksichtigt. 

Beispiel: \emph{IBAN\_Laenge}

\subsubsection{Fehlertext}
In diesem Feld wird der Text eingetragen, der dem Anwender angezeigt werden soll. Meistens wird genau der Text dem Anwender angezeigt. Manchmal werden Platzhalter verwendet. Wie diese zu ersetzen sind, geht meistens aus der Beschreibung hervor.

Beispiel: \emph{Die von Ihnen eingegebene IBAN hat nicht die erforderlichen 22 Stellen.}

\subsubsection{Beschreibung}
Die Beschreibung enth�lt die Logik, die bestimmt, unter welchen Bedingungen die Meldung ausgegeben werden soll. Die Logik wird in nat�rlicher Sprache beschrieben. Die Qualit�t der Beschreibungen variiert stark -- abh�ngig von der Komplexit�t der Pr�fung und dem Hintergrund des Autors.

In der Beschreibung k�nnen Felder aus den Verkn�pften Formularen referenziert werden. Wenn das gemacht wird, kann die Referenz technisch zu einem Feld im Formular zugeordnet werden. In manchen F�llen wird nur der technische Bezeichner eines Formularfeldes (wie \emph{bi\_iban} im unten stehenden Beispiel) oder nur eine nat�rlichsprachliche Umschreibung (bspw. \glqq{}In der ersten Spalte der Tabelle\grqq{}) verwendet.

Beispiel \emph{bi\_iban: Wenn es sich um eine DE - IBAN handelt, muss sie incl. DE genau 22 Zeichen lang sein.}

\subsubsection{Fehlerlevel}
Das Fehlerlevel ist der Schweregrad der Pr�fung. Es reicht von Hinweis bis Disqualifikation. Es wird �ber ein Auswahlfeld im Formularverwalter erfasst. Es ist dadurch eindeutig und technisch auswertbar. Es gibt 5 Level:
\begin{description}
	\item[Information, Warnung, Fehler] Diese Level sind rein informativ. Die Meldungen werden angezeigt, schr�nken die Bearbeitung aber nicht weiter ein.
	\item[Fataler Fehler] Enth�lt der Antrag ein Formular mit einem fatalen Fehler, kann er nicht eingereicht werden.
	\item[Disqualifikation] Ein Disqualifikationsfehler schlie�t ein Formular vom Antrag aus. Der Antrag kann eingereicht werden, aber das entsprechende Formular wird nicht mit eingereicht.
\end{description}

Beispiel: \emph{Fataler Fehler}

\subsubsection{verkn�pfte Formulare und Antragsverfahren}
Eine Plausi-Vorgabe ist mit den Antragsverfahren verkn�pft, f�r die sie gelten soll. Und sie ist mit den Formularen Verkn�pft, auf die f�r die Pr�fung zugegriffen werden muss. 

Beispiel: \\
\emph{Antragsverfahren: Brandenburg Agrar-Antrag 2026, Brandenburg ELER-Antrag 2026, Mecklemburg-Vorpommern Agrar-Antrag 2026, Mecklemburg-Vorpommern ELER-Antrag 2026, Schleswig Holstein Agrar-Antrag 2026}\\
\emph{Dokument: Stammdaten}

\section{Implementierung}
Diese Vorgaben werden von der data experts mit Hilfe der propriet�ren Skriptsprache \glqq{}formScript\grqq{} implementiert. Diese ist f�r die Verarbeitung und Manipulation von Formulardaten geschaffen worden. Diese Skripte werden als \gls{xml}-Dateien im Quellcode hinterlegt. Zur Compile-Zeit werden sie zu Java-Code �bersetzt. \ocLstref{iban-laenge} ist das Skript zu der Vorgabe \glqq{}IBAN\_Laenge\grqq{}, die oben als Beispiel verwendet wurde.

\begin{lstlisting}[language=xml, caption={IBAN\_Laenge Skript}, label=iban-laenge]
<?xml version="1.0" encoding="UTF-8"?><scripts>
<script name="MC26_STDA_1_IBAN_Laenge"><![CDATA[
@meta name: "MC26_STDA_1_IBAN_Laenge";

master stda1 as #202617101;

var iban = stda1->bi_IBAN@value;
if(Strings.startsWith(iban, "DE") 
   && Strings.length(iban) != 22) {
	fatal [stda1->bi_IBAN]
	: "Die von Ihnen eingegebene IBAN hat nicht die 
		erforderlichen 22 Stellen."
	: meta {errorId: "IBAN_Laenge"};
}]]></script>
</scripts>
\end{lstlisting}