\chapter{Ziele}
\plannedpages{3 - 6 | T: 23.11.25}
\note{
Was sind die operativen Ziele der Arbeit (z. B. Prototyp, Evaluationskriterien, Metriken)?

Welche forschungsleitende Hypothese(n) werden getestet?

Welche messbaren Erfolgskriterien definierst du (z. B. Genauigkeit, Zeitersparnis, Verst�ndlichkeit, Fehlerreduktion)?

Welche Nebenbedingungen gelten (z. B. verwendete DSL, Datengrundlage, Datenschutz)?

Welche Annahmen werden explizit getroffen und wie werden sie �berpr�ft?
}

Im Zentrum dieser Arbeit steht das Generieren eines ver�nderten Skriptes. Auf Basis einer angepassten Kundenvorgabe.
Daf�r wird explorativ untersucht, mit welchen \gls{ki}-Ans�tzen die �nderung der Kundenvorgaben am pr�zisesten in Skripte �bersetzt werden k�nnen. 
 
Dabei werden verschiedene Modelle, Prompting-Methoden und Kontextinformationen getestet, um die Auswirkungen auf Kompilierbarkeit, Testfall-Erfolg und Codequalit�t zu bewerten.

\section{Forschungsziele}
Welche Konfiguration aus Prompting-Methode, KI-Modell und Kontextinformationen (\glqq{}Prototyp\grqq{}) erzielt bei der Anpassung von Skripten auf Basis von Kundenvorgaben die h�chste syntaktische und semantische Korrektheit?

Zur Beantwortung der Forschungsfrage sollen folgende Hypothesen �berpr�ft werden:
\begin{enumerate}
	\item Teurere OpenAI Modelle, liefern bessere syntaktische und semantische Korrektheit sowie geringere Komplexit�t als g�nstigere.
	\item Mehr Beispiele im Kontext erh�hen die syntaktische Korrektheit.
	\item Relevante Formulardefinitionen im Kontext erh�hen die semantische Korrektheit.
	\item API und JDOC im Kontext erh�hen die syntaktische und semantische Korrektheit und verringern die Komplexit�t.
	\item Chain-of-Thought-Prompting verbessert die Korrektheit von non-reasoning Modellen.
\end{enumerate}

\section{Operative Ziele}
In der Arbeit sollen drei Artefakte entstehen: Testdatensatz-Sammlung, Evaluations-Pipeline und Prototypen.
\subsection{Testdatensatz-Sammlung}
 Es soll eine Testdatensatz-Sammlung entstehen, die mindestens 30 \todo{Quelle raussuchen, warum 30 die magische Untergrenze ist} hat. Ein Testdatensatz in dieser Sammlung soll enthalten:
	\begin{itemize}
		\item Die Kundenvorgabe vor der Anpassung
		\item Die Kundenvorgabe nach der Anpassung
		\item Das Skript vor der Anpassung
		\item Das Skript nach der Anpassung (Referenzwahrheit)
		\item Teststories, die das Skript nach der Anpassung validieren
	\end{itemize}
	
\subsection{Evaluations-Pipeline}
Diese Pipeline soll f�r jeden Eintrag in der Testdatensatz-Sammlung mit einer Prompting-Methode von einem KI-Modell das \glqq{}Skript nach der Anpassung generieren\grqq{} lassen. Die generierten Skripte sollen evaluiert und ein Bericht �ber die Ergebnisse erstellt werden.

Der Bericht soll folgende Metriken enthalten:
\begin{description}
	\item[Syntaktische Korrektheit] Ein Skript ist syntaktisch korrekt, wenn es kompilierbar ist.
	\item[Semantische Korrektheit] Ein Skript ist semantisch korrekt, wenn es alle Testf�lle erf�llt.
	\item[Komplexit�t] Komplexit�t des Skriptes im Verglich zur Referenzwahrheit.
	\item[Kosten] Preis (in Euro) f�r die Generierung eines Skriptes.
\end{description}

\subsection{Prototypen}
Ein Tripel aus Prompting-Methode, KI-Modell und Kontext-Information stellt einen Prototypen dar.

Im folgenden sind die Varianten der einzelnen Bestandteile aufgelistet, die zu Prototypen kombiniert werden sollen:
\subsubsection{Prompting-Methoden}
	\begin{itemize}
		\item Zero-Shot-Prompting
		\item One-Shot-Prompting
		\item Few-Shot-Prompting
		\item Chain-of-Thought-Prompting
	\end{itemize}
\subsubsection{KI-Modelle}
	\begin{itemize}
		\item GPT-5-nano (kleines Reasoning-Modell)
		\item GPT-5.1 (gro�es Reasoning-Modell)
		\item GPT-5.1-Codex (Modell spezialisiert auf agentic coding)
		\item GPT-4.1 (\glqq{}Kl�gstes nicht-reasoning-Modell\grqq{})
	\end{itemize}
\subsubsection{Kontext-Informationen}
	\begin{itemize}
		\item Keine weiteren Kontextinformationen
		\item Formulardefinition der relevanten Formulare
		\item API und JDOC der zur Verf�gung stehenden Funktionen
	\end{itemize}


\section{Annahmen und Nebenbedingungen}
Die \gls{deg} stellt zur Durchf�hrung dieser Arbeit Zugang und Kapital f�r die Verwendung von OpenAI Modellen bereit. Deshalb werden diese verwendet.

Die Skripte werden in der dom�nenspezifischen Sprache \glqq{}formScript\grqq{} umgesetzt. Diese ist au�erhalb der \gls{deg} unbekannt und dadurch nicht in den Trainingsdaten der KI-Modelle vorhanden. Sie wird bis 2028 verwendet werden. Es ist geplant, dass die Skripte im Jahr 2028 im Rahmen einer technologischen Modernisierung in die Skriptsprache TypeScript migriert werden. Es wird davon ausgegangen, dass das Generieren syntaktisch korrekter Skripte dadurch einfacher wird und die semantische Korrektheit eine Herausforderung bleibt.

Als Basis zur Erstellung der Testdatensatz-Sammlung wird der aktuelle Daten zugegriffen. Vorgaben und ihre �nderungen werden per API aus dem produktiven System des Formularverwalters abgerufen. Skripte und ihre �nderungen werden aus der Versionskontrolle des Sourcecode-Repositories des Webclients entnommen. Es wird davon ausgegangen, dass sich aus diesen Datenbest�nden ausreichend Testdatens�tze extrahieren lassen. Daf�r wird von drei Voraussetzungen ausgegangen:\\
\begin{enumerate}
	\item Die Zuordnung von Formularverwalter-Vorgaben zu Webclient-Skripten ist in ausreichender Qualit�t und Quantit�t m�glich.
	\item Es liegen ausreichend Skripte im Webclient vor, f�r die Teststories definiert sind.
	\item Die Teststories sind von ausreichender Qualit�t, sodass das Erf�llen der Teststories ein semantisch korrektes Skript bedeutet.
\end{enumerate}


