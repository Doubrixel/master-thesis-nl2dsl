%%%%%%%%%%%%%%%%%%%%%%%%%%%%%%
%Dokumentenklasse definieren%%
%%%%%%%%%%%%%%%%%%%%%%%%%%%%%%
\documentclass[%
	%draft,     				% Entwurfsstadium
  final,      				% fertiges Dokument
	%%%% --- Schriftgröße ---
  12pt,
  bigheadings,      	% große Überschriften
	%%%% --- Sprache ---
  ngerman,          	% wird an andere Pakete weitergereicht
	%%%% === Seitengröße ===
  a4paper,
	%%%% === Optionen für den Satzspiegel ===
  %BCOR15mm,          	% Zusaetzlicher Rand auf der Innenseite
  DIV14,            	% Seitengroesse (siehe Koma Skript Dokumentation(DIVcalc)!)
  1.1headlines,     	% Zeilenanzahl der Kopfzeilen
  headexclude,      	% Kopf nicht einbeziehen
  footexclude,      	% Fuss nicht einbeziehen
  mpexclude,        	% Margin nicht einbeziehen
  pagesize,         	% Schreibt die Papiergroesse in die Datei.
                    	% Wichtig fuer Konvertierungen
	%%%% === Layout ===
  oneside,         	% einseitiges Layout
  %twoside,          	% Seitenraender für zweiseitiges Layout
  %onecolumn,        	% Einspaltig
  openright,       		% Kapitel beginnen immer auf der rechten Seite
                   		% (macht nur bei 'twoside' Sinn)
  titlepage,       		% Titel als einzelne Seite ('titlepage' Umgebung)
	%%%% --- Absatzeinzug ---
  %                 	% Absatzabstand: Einzeilig,
  %parskip,         	% Freiraum in letzter Zeile: 1em
  %                 	% Absatzabstand: Halbzeilig
  halfparskip,     		% Freiraum in letzter Zeile: 1em
  %                 	% Absatzabstand: keiner
  %parindent,        	% Eingerückt (Standard)
	%%%% --- Kolumnentitel ---
  headsepline,     	 	% Linie unter Kolumnentitel
  %headnosepline,  	 	% keine Linie unter Kolumnentitel
  footsepline,				% Linie unter Fussnote
  %footnosepline,  	 	% keine Linie unter Fussnote
	%%%% --- Kapitel ---
  nochapterprefix,  	% keine Ausgabe von 'Kapitel:'
	%%%% === Verzeichnisse (TOC, LOF, LOT, BIB) ===
  liststotoc,      		% Tabellen & Abbildungsverzeichnis ins TOC
  idxtotoc,        		% Index ins TOC
  bibtotoc,         	% Bibliographie ins TOC
  %bibtotocnumbered, 	% Bibliographie im TOC nummeriert
  %liststotocnumbered,% Alle Verzeichnisse im TOC nummeriert
  tocindent,        	% eingereuckte Gliederung
  %tocleft,         	% Tabellenartige TOC
  listsindent,      	% eingereuckte LOT, LOF
  %listsleft,       	% Tabellenartige LOT, LOF
  %pointednumbers,  	% Überschriftnummerierung mit Punkt, siehe DUDEN !
  pointlessnumbers, 	% Überschriftnummerierung ohne Punkt, siehe DUDEN !
  %openbib,         	% alternative Formatierung des Literaturverzeichnisses
	%%%% === Matheformeln ===
  %leqno,           	% Formelnummern links
  fleqn,            	% Formeln werden linksbuendig angezeigt
] {scrbook}						% Klassen: scrartcl, scrreprt, scrbook
%%%%%%%%%%%%%%%%%%%%%%%%%%%%%%
%Einbinden von neuen Paketen%%
%%%%%%%%%%%%%%%%%%%%%%%%%%%%%%
\areaset{15cm}{25cm}
%\setlength{\voffset}{+1.5cm}
%Zuschneiden von Schriftarten (muss am Anfang stehen)
\usepackage{fix-cm}

\usepackage{moreverb}

\usepackage{stmaryrd}

%\usepackage{floatflt}

\usepackage[dvips]{epsfig}
\newcommand{\clearemptydoublepage}{%                 % neue Kapitel auf ungerade Seite
  \newpage{\pagestyle{empty}%
  \cleardoublepage}}
  
%Stilvorlage fürs Literaturverzeichnis
%\usepackage{harvard}
%\usepackage[numbers]{natbib}

\usepackage[round]{natbib}
%\usepackage[numbers]{natbib}

%\usepackage{har2nat}
%\usepackage{bibgerm}
%\usepackage{jurabib}

%Web Support for BiTex
\usepackage{xurl}

%Das folgende Packet hyperref führt dazu,
%dass %die wichtigsten Dokumenteneigenschaften in der PDF-Datei eingetragen werden
%(URL, Titel, Autor, Kurzbeschreibung und Stichwörter),
%das Inhaltsverzeichnis und die Fußnoten verlinkt werden,
%URLs verlinkt werden und
%URLs im Text dargestellt und umgebrochen werden können.
\usepackage[
	bookmarks=true,								% Lesezeichen erzeugen
	bookmarksopen=false,					% Lesezeichen ausgeklappt
	bookmarksnumbered=true,				% Anzeige der Kapitelzahlen am Anfang der Namen der Lesezeichen
	pdfstartpage=1,							% Seite, welche automatisch geöffnet werden soll
	%baseurl=http://www.server.de/dateiname.pdf, 
	% URL des PDF-Dokuments (oder Hintergrundinformationen)
	pdftitle={KI-gestützte Übersetzung natürlichsprachlicher Prüfungslogik in DSL-Skripte am Beispiel von Plausibilitätsprüfungen für digitale Antragsformulare},
  % Titel des PDF-Dokuments
	pdfauthor={Lemnitzer, Florian},	% Autor(Innen) des PDF-Dokuments
	pdfsubject={},	% Inhaltsbeschreibung des
	pdfkeywords={},
  % Stichwortangabe zum PDF-Dokument
	breaklinks=true,							% ermöglicht einen Umbruch von URLs
	colorlinks=true,							% Einfärbung von Links
	linkcolor=black,							% Linkfarbe: blau
	anchorcolor=blue,						% Ankerfarbe: schwarz
	citecolor=black, 							% Literaturlinks: schwarz
	filecolor=black,							% Links zu lokalen Dateien: schwarz
	menucolor=black, 							% Acrobat Menü Einträge: schwarz
	pagecolor=black, 							% Links zu anderen Seiten im Text: schwarz
	urlcolor=black,							% URL-Farbe: blau
	%backref=true,
	pagebackref=false,
	pdfcenterwindow=true,
	pdfnewwindow=true,
	pdffitwindow=true,
	pdfstartview=FitH,
	pdfpagemode=UseOutlines
] {hyperref}


% für wichtig-box ..



%\newenvironment{important}{
%{\color{black}\ovalbox{
%\parbox{428}{\textcolor{black}{
%\begin{center}
%\begin{minipage}{5.5 in}
%\large{\Pointinghand} \normalsize 
%}
%{
%\end{minipage}
%\end{center}
%}}}}
%}


%Silbentrennung nach neuer deutschen Rechtschreibung
%\usepackage[ngerman]{babel}
\usepackage[german]{babel}

%T1 Zeichensatzkodierung
\usepackage[T1]{fontenc}

%Sonderzeichen der deutschen Sprache z.B. ß
\usepackage[ansinew]{inputenc}

%%%%%%%%%%%%%
%%Schriften%%
%%%%%%%%%%%%%

%Schrift ändern
%\renewcommand{\familydefault}{pag}

%Schriftarten ändern
\newcommand{\changefont}[3]{\fontfamily{#1} \fontseries{#2} \fontshape{#3} \selectfont}

% Latin Modern
%\usepackage{lmodern}
% -------------------
% Palantino , Helvetica, Courier
\usepackage{mathpazo}
\usepackage[scaled=.95]{helvet}
\usepackage{courier}

% Erlaubt automatische Trennung von Worten mit Umlauten
%\usepackage{ae}

%Zum definieren der einzelnen Elemente im Text. Sollte aber beim Standard belassen werden
\usepackage{typearea}

%Zum definieren der Zeilenabstandes
\usepackage{setspace}

%durch dieses Paket ist das Einbinden von Grafiken möglich, aber nur eps Dateien
\usepackage[]{graphicx}

%durch dies Pakete können Bilder mit Textumflossen werden
%\usepackage{floatflt}
%\usepackage{/picins}

%durch dieses Paket werden die Farbnamen definiert

\usepackage[usenames]{color}


%hierdurch wird Quellcode besser dargestellt
\usepackage{listings}
\renewcommand\lstlistlistingname{Quellcodeverzeichnis}
\renewcommand{\lstlistingname}{Quellcode}
\usepackage{xcolor}

%Abkürzungsverzeichnis
\usepackage[intoc]{nomencl}
\usepackage[normalem]{ulem} %Möglichkeiten zum welligen Unterstreichen bzw. durchstreichen von Text

\usepackage[acronym, toc,languages=german]{glossaries}
\newglossaryentry{signavio}{
	name=Signavio,
	description=ist eine Business-Process-Management-Software von SAP \cite[vgl.][]{sapExplorerSignavio}
}
\newglossaryentry{process-mining}{
	name=Process-Mining,
	description=ist eine wissenschaftliche Disziplin zwischen Data-Mining und Prozessmodellierung. Ziel ist das Entdecken\, die �berwachung und die Verbesserung echter Prozesse \cite[vgl.][S. 1]{westergaardProcessMiningManifesto}
}
\newglossaryentry{jira}{
	name=Jira,
	description=ist eine \glqq{}Software f�r die Vorgangs- und Projektverfolgung\grqq{} \cite{atlassianJiraSoftwareFur}
}
\newglossaryentry{celonis}{
	name=Celonis,
	description=ist eine \gls{process-mining}-Software vom gleichnamigen Unternehmen \cite[vgl.][]{celonisCelonisProcessAnalytics}
}
\newglossaryentry{confluence}{
	name=Confluence,
	description=ist eine Wiki-Software \cite[vgl.][]{altlassianConfluenceRemotefreundlicheArbeitsbereich}
}
\newglossaryentry{gitlab}{
	name=GitLab,
	description=ist eine Online-Dienst zur Versionsverwaltung von Softwareprojekten \cite[vgl.][]{gitlabDevSecOpsPlatform}
}
\newglossaryentry{profil}{
	name=profil,
	description=ist ein Gesch�ftsbereich der \gls{deg}\, welcher L�sungen zur F�rdermittelverwaltung vermarktet \cite[vgl.][]{dataexpertsgmbhProfilDataExperts}
}


\newacronym{deg}{deg}{data experts gmbh}
\newacronym{swot}{SWOT-Analyse}{St�rken, Schw�chen, Chancen, Risiken -- Analyse}
\newacronym{vrinos}{VRINOS-Analyse}{Value, Rarity, Imitablility, Non-Substitutability, Organization, Sustainability -- Analyse}
\newacronym{ag}{AG}{Arbeitsgemeinschaft}
\newacronym{iso}{ISO}{International Organization for Standardization}
\newacronym{bpmn}{BPMN}{Business Process Model and Notation}
\newacronym{api}{API}{Application Programming Interface}
\newacronym{csv}{CSV}{Comma-separated Values}
\newacronym{pg}{PG}{Projektgruppe}
\newacronym{qs}{QS}{Qualit�tssicherung}
\newacronym{dmn}{DMN}{Decision Model and Notation}
\newacronym{hkr}{HKR}{Haushalts-, Kassen- und Rechnungswesen}
\newacronym[longplural={geografischen Informationssystem}, shortplural={GIS}]{gis}{GIS}{geografisches Informationssystem}
\newacronym[longplural={dom�nenspezifischen Sprache}, shortplural={DSL}]{dsl}{DSL}{dom�nenspezifische Sprache}
\newacronym{rfk}{RFK}{Referenzfl�chenkataster}
\newacronym{gb}{GB}{Gesch�ftsbereich}
\newacronym{bmel}{BMEL}{Bundesministerium f�r Ern�hrung und Landwirtschaft}
\newacronym{gap}{GAP}{gemeinsame Agrarpolitik}
\newacronym{eu}{EU}{europ�ische Union}
\newacronym{ziaf}{ZIAF}{Zahlstelle InVeKoS Agrarf�rderung}
\newacronym{invekos}{InVeKoS}{integriertes Verwaltungs- und Kontrollsystem}
\newacronym{b2b}{B2B}{Business to Business}
\newacronym{b2c}{B2C}{Business to Customer}
\newacronym{ki}{KI}{k�nstliche Intelligenz}
\newacronym{crm}{CRM}{Customer Relationship Management}
\newacronym{xml}{XML}{Extended Markup Language}
\newacronym{wysiwyg}{WYSIWYG}{\glqq{}What you see is what you get\grqq{}}

\makeglossaries
%%%%%%%%%%%%%%%%%%%%
%%Allgemeine Dinge%%
%%%%%%%%%%%%%%%%%%%%

%Zeilenabstand 1.5
\onehalfspace 

%%%%%%%%%%%%%%%%%%%%%%%%%%%%%%%%%%%%%%%%%%
%%Definieren des Abkürzungsverzeichnises%%
%%%%%%%%%%%%%%%%%%%%%%%%%%%%%%%%%%%%%%%%%%

% Befehl umbenennen in abk
\let\abk\nomenclature

% Befehl damit in der Kopfzeile auf der zweiten Seite auch Abkürzungsverzeichnis angezeigt wird
\newcommand{\Abkuerzungsverzeichnis}{
%\clearpage % bei Option "oneside"
\cleardoublepage % bei Option "twoside"
\markboth{\nomname}{\nomname}
\printnomenclature
\newpage
}

% Deutsche Überschrift
\renewcommand{\nomname}{Abkürzungsverzeichnis}

% Punkte zw. Abkürzung und Erklärung
\setlength{\nomlabelwidth}{.20\hsize}
\renewcommand{\nomlabel}[1]{#1 \dotfill}

% Zeilenabstände verkleinern
\setlength{\nomitemsep}{-\parsep}
\makenomenclature

%Unterstreichung des Markup Buchstaben
\newcommand{\markup}[1]{\uline{#1}}

%Optionale Argument können mit dem voreingestellten Wert verglichen werden
\usepackage{ifthen}

%%%%%%%%%%%%%%%%%%%%%%%%%%%%%%%%%%%%%%
%%Definieren der Kopf- und Fußzeilen%%
%%%%%%%%%%%%%%%%%%%%%%%%%%%%%%%%%%%%%%

\usepackage{fancyhdr} 			%Paket laden
\pagestyle{fancy}						%eigener Seitenstil
\fancyhf{}									%alle Kopf- und Fußzeilenfelder bereinigen
\fancyhead[OR]{						 	% "O" steht für "odd", also ungerade Seiten; "L" steht für links
	\color{black}
	\changefont{pag}{m}{n}
	\bfseries\leftmark
}
\fancyhead[EL]{ % "E" für "even", also gerade Seiten; "R" steht für rechts
%\fancyhead[L]{
	\color{black}
	\changefont{pag}{m}{n}
	\bfseries\rightmark
}
\renewcommand{\chaptermark}[1]{\markboth{#1}{}}
\renewcommand{\sectionmark}[1]{\markright{\thesection\ #1}}
\renewcommand{\headrulewidth}{0.4pt}				%obere Trennlinie

\fancyfoot[OR,EL]{			%Seitennummer
%\fancyfoot[OR,L]{			%Seitennummer
	\color{black}
	\changefont{pag}{m}{n}
	\bfseries\thepage
}	
\renewcommand{\footrulewidth}{0.4pt}				%untere Trennlinie

%Plain umdefinieren für z.B. neues Kapitel
\fancypagestyle{plain}{
	\fancyhead{} 												
	\renewcommand{\headrulewidth}{0pt}
	\renewcommand{\footrulewidth}{0.4 pt}
}

%Farbe für Trennlinie im Kopf wechseln
\renewcommand{\headrule}{{\color{black}
	\hrule width\headwidth height\headrulewidth \vskip-\headrulewidth}
}

%Farbe für Trennlinie im Fuß wechseln
\renewcommand{\footrule}{{\color{black}
	\hrule width\headwidth height\headrulewidth \vskip-\headrulewidth \vspace{2mm}}
}

%zum definieren von Farben
\definecolor{myBlue} {rgb}{0.423529412,0.388235294,0.725490196}
\definecolor{myRed} {cmyk}{0.02,0.88,0.99,0.0}
\definecolor{myBlue2} {cmyk}{0.681,0.511,0,0.471}
\definecolor{myTablehead} {cmyk}{0.57,0.19,0,0}

%%%%
%%TABELLEN
%%%%

\usepackage{colortbl}
\usepackage{hhline,float}
\usepackage{array}
\usepackage{booktabs} %einzelne Linien in Tabellen dicker oder dünner
\usepackage{supertabular} %Tabellen über mehrere Seiten

%%%%
%%Tabellen vorbereiten
%%
%%Parameter
%%   Caption
%%   Label
%%   Table
%%%%
\newcommand{\ocTable}[4]{
	\begin{center}
		\begin{table}[htbp]
			#3
			\caption{#1}
			\label{tbl:#2}
			#4
		\end{table}
	\end{center}
}

%%%%
%%Tabellenkopf vorbereiten
%%%%
\newcommand{\ocTableHeadRow}{\rowcolor{myTablehead}}
\newcommand{\ocTablHead}[1]{
	\textcolor{white}{\textbf{#1}}
}
%%%%
%%Tabellensatzformatierungen von Herrn Voß
%%http://www.torsten-schuetze.de/tex/tabsatz-2004.pdf
%%%%
\newcolumntype{f}{>{$}l<{$}}
\newcolumntype{n}{l}
\newcolumntype{N}{>{\scriptsize}l}
\newcolumntype{v}[1]{>{\raggedright\hspace{0pt}}p{#1}}
\newcolumntype{V}[1]{>{\scriptsize\raggedright\hspace{0pt}}p{#1}}
\newcolumntype{B}[1]{>{\boldmath\DC@{.}{,}{#1}}l<{\DC@end}}
\newcolumntype{d}[1]{>{\DC@{.}{,}{#1}}l<{\DC@end}}
\newcolumntype{R}[1]{%
	>{\begin{turn}{90}\begin{minipage}{#1}\scriptsize\raggedright\hspace{0pt}}l%
	<{\end{minipage}\end{turn}}%
}
\newcolumntype{x}{>{\scriptsize\raggedright\hspace{0pt}}X}
\newcolumntype{A}{>{\columncolor{myTablehead}}c}

%%%%
%%Seitenverweise
%%%%
% wenn gleiche Seite dann keinen Seitenverweis
\newcommand{\ocRef}[2]{#1 \ref{#2}\label{label #2}\ifthenelse{\equal{\pageref{label #2}}{\pageref{#2}}}{}{, S. \pageref{#2}}}
\newcommand{\ocFigref}[1]{\ocRef{Abb.}{fig:#1}}
\newcommand{\ocTblref}[1]{\ocRef{Tabelle}{tbl:#1}}
\newcommand{\ocPararef}[1]{\ocRef{Abschnitt}{sec:#1}}
\newcommand{\ocChapterref}[1]{\ocRef{Kapitel}{sec:#1}}
\newcommand{\ocLstref}[1]{\ocRef{Quellcode}{#1}}
%\newcommand{\ocNameref}[1]{`\nameref{#1}' auf Seite~\pageref{#1}}

%To-Do Befehl
\newcommand{\todo}[1]{\textbf{\textsc{\textcolor{red}{(TODO: #1)\\}}}}
\newcommand{\vera}[1]{\textbf{\textsc{\textcolor{purple}{(Frage an Vera: #1)\\}}}}
\newcommand{\niels}[1]{\textbf{\textsc{\textcolor{blue}{(Frage an Niels: #1)\\}}}}
\newcommand{\plannedpages}[1]{\textbf{\textsc{\textcolor{green}{(Geplante Seitenzahl: #1)\\}}}}
\newcommand{\reviewng}[1]{\textbf{\textsc{\textcolor{orange}{(Anmerkung Niels: #1)}}}}
\newcommand{\reviewvm}[1]{\textbf{\textsc{\textcolor{orange}{(Anmerkung Vera: #1)}}}}

\newboolean{shownotes}
\setboolean{shownotes}{true}

\newcommand{\note}[1]{%
	\ifthenelse{\boolean{shownotes}}{%
		{\colorbox{gray!10}{%
			\parbox{1\linewidth}{	
					\linespread{1}\selectfont   % etwas mehr Zeilenabstand
					\setlength{\parskip}{6pt}
					\ttfamily 
					Leitfragen
					
					\footnotesize
					#1
				}%
			}%
		}%
	}{}%
}

\newcommand{\realnote}[1]{%
	\ifthenelse{\boolean{shownotes}}{%
		{\colorbox{gray!10}{%
				\parbox{1\linewidth}{	
					\linespread{1}\selectfont   % etwas mehr Zeilenabstand
					\setlength{\parskip}{6pt}
					\ttfamily 					
					\footnotesize
					#1
				}%
			}%
		}%
	}{}%
}

\newcommand{\beispiel}[1]{%
	\noindent
	\setlength{\fboxsep}{6pt} % Abstand zwischen Text und Rahmen
	\fcolorbox{black}{gray!10}{%
		\parbox{\dimexpr\linewidth-2\fboxsep-2\fboxrule}{%
			\textbf{Beispiel:} #1
		}%
	}%
}

% Benutzung:
%\beispiel{Dies ist ein Beispieltext ohne zusätzliche Pakete.}

\setlength\fboxsep{0.5pt}
\setlength\fboxrule{0.3pt}
