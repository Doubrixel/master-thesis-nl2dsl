% einstellungen
%%%%%%%%%%%%%%%%%%%%%%%%%%%%%%
%Dokumentenklasse definieren%%
%%%%%%%%%%%%%%%%%%%%%%%%%%%%%%
\documentclass[%
	%draft,     				% Entwurfsstadium
  final,      				% fertiges Dokument
	%%%% --- Schriftgröße ---
  12pt,
  bigheadings,      	% große Überschriften
	%%%% --- Sprache ---
  ngerman,          	% wird an andere Pakete weitergereicht
	%%%% === Seitengröße ===
  a4paper,
	%%%% === Optionen für den Satzspiegel ===
  %BCOR15mm,          	% Zusaetzlicher Rand auf der Innenseite
  DIV14,            	% Seitengroesse (siehe Koma Skript Dokumentation(DIVcalc)!)
  1.1headlines,     	% Zeilenanzahl der Kopfzeilen
  headexclude,      	% Kopf nicht einbeziehen
  footexclude,      	% Fuss nicht einbeziehen
  mpexclude,        	% Margin nicht einbeziehen
  pagesize,         	% Schreibt die Papiergroesse in die Datei.
                    	% Wichtig fuer Konvertierungen
	%%%% === Layout ===
  oneside,         	% einseitiges Layout
  %twoside,          	% Seitenraender für zweiseitiges Layout
  %onecolumn,        	% Einspaltig
  openright,       		% Kapitel beginnen immer auf der rechten Seite
                   		% (macht nur bei 'twoside' Sinn)
  titlepage,       		% Titel als einzelne Seite ('titlepage' Umgebung)
	%%%% --- Absatzeinzug ---
  %                 	% Absatzabstand: Einzeilig,
  %parskip,         	% Freiraum in letzter Zeile: 1em
  %                 	% Absatzabstand: Halbzeilig
  halfparskip,     		% Freiraum in letzter Zeile: 1em
  %                 	% Absatzabstand: keiner
  %parindent,        	% Eingerückt (Standard)
	%%%% --- Kolumnentitel ---
  headsepline,     	 	% Linie unter Kolumnentitel
  %headnosepline,  	 	% keine Linie unter Kolumnentitel
  footsepline,				% Linie unter Fussnote
  %footnosepline,  	 	% keine Linie unter Fussnote
	%%%% --- Kapitel ---
  nochapterprefix,  	% keine Ausgabe von 'Kapitel:'
	%%%% === Verzeichnisse (TOC, LOF, LOT, BIB) ===
  liststotoc,      		% Tabellen & Abbildungsverzeichnis ins TOC
  idxtotoc,        		% Index ins TOC
  bibtotoc,         	% Bibliographie ins TOC
  %bibtotocnumbered, 	% Bibliographie im TOC nummeriert
  %liststotocnumbered,% Alle Verzeichnisse im TOC nummeriert
  tocindent,        	% eingereuckte Gliederung
  %tocleft,         	% Tabellenartige TOC
  listsindent,      	% eingereuckte LOT, LOF
  %listsleft,       	% Tabellenartige LOT, LOF
  %pointednumbers,  	% Überschriftnummerierung mit Punkt, siehe DUDEN !
  pointlessnumbers, 	% Überschriftnummerierung ohne Punkt, siehe DUDEN !
  %openbib,         	% alternative Formatierung des Literaturverzeichnisses
	%%%% === Matheformeln ===
  %leqno,           	% Formelnummern links
  fleqn,            	% Formeln werden linksbuendig angezeigt
] {scrbook}						% Klassen: scrartcl, scrreprt, scrbook
%%%%%%%%%%%%%%%%%%%%%%%%%%%%%%
%Einbinden von neuen Paketen%%
%%%%%%%%%%%%%%%%%%%%%%%%%%%%%%
\areaset{15cm}{25cm}
%\setlength{\voffset}{+1.5cm}
%Zuschneiden von Schriftarten (muss am Anfang stehen)
\usepackage{fix-cm}

\usepackage{moreverb}

\usepackage{stmaryrd}

%\usepackage{floatflt}

\usepackage[dvips]{epsfig}
\newcommand{\clearemptydoublepage}{%                 % neue Kapitel auf ungerade Seite
  \newpage{\pagestyle{empty}%
  \cleardoublepage}}
  
%Stilvorlage fürs Literaturverzeichnis
%\usepackage{harvard}
%\usepackage[numbers]{natbib}

\usepackage[round]{natbib}
%\usepackage[numbers]{natbib}

%\usepackage{har2nat}
%\usepackage{bibgerm}
%\usepackage{jurabib}

%Web Support for BiTex
\usepackage{xurl}

%Das folgende Packet hyperref führt dazu,
%dass %die wichtigsten Dokumenteneigenschaften in der PDF-Datei eingetragen werden
%(URL, Titel, Autor, Kurzbeschreibung und Stichwörter),
%das Inhaltsverzeichnis und die Fußnoten verlinkt werden,
%URLs verlinkt werden und
%URLs im Text dargestellt und umgebrochen werden können.
\usepackage[
	bookmarks=true,								% Lesezeichen erzeugen
	bookmarksopen=false,					% Lesezeichen ausgeklappt
	bookmarksnumbered=true,				% Anzeige der Kapitelzahlen am Anfang der Namen der Lesezeichen
	pdfstartpage=1,							% Seite, welche automatisch geöffnet werden soll
	%baseurl=http://www.server.de/dateiname.pdf, 
	% URL des PDF-Dokuments (oder Hintergrundinformationen)
	pdftitle={KI-gestützte Übersetzung natürlichsprachlicher Prüfungslogik in DSL-Skripte am Beispiel von Plausibilitätsprüfungen für digitale Antragsformulare},
  % Titel des PDF-Dokuments
	pdfauthor={Lemnitzer, Florian},	% Autor(Innen) des PDF-Dokuments
	pdfsubject={},	% Inhaltsbeschreibung des
	pdfkeywords={},
  % Stichwortangabe zum PDF-Dokument
	breaklinks=true,							% ermöglicht einen Umbruch von URLs
	colorlinks=true,							% Einfärbung von Links
	linkcolor=black,							% Linkfarbe: blau
	anchorcolor=blue,						% Ankerfarbe: schwarz
	citecolor=black, 							% Literaturlinks: schwarz
	filecolor=black,							% Links zu lokalen Dateien: schwarz
	menucolor=black, 							% Acrobat Menü Einträge: schwarz
	pagecolor=black, 							% Links zu anderen Seiten im Text: schwarz
	urlcolor=black,							% URL-Farbe: blau
	%backref=true,
	pagebackref=false,
	pdfcenterwindow=true,
	pdfnewwindow=true,
	pdffitwindow=true,
	pdfstartview=FitH,
	pdfpagemode=UseOutlines
] {hyperref}


% für wichtig-box ..



%\newenvironment{important}{
%{\color{black}\ovalbox{
%\parbox{428}{\textcolor{black}{
%\begin{center}
%\begin{minipage}{5.5 in}
%\large{\Pointinghand} \normalsize 
%}
%{
%\end{minipage}
%\end{center}
%}}}}
%}


%Silbentrennung nach neuer deutschen Rechtschreibung
%\usepackage[ngerman]{babel}
\usepackage[german]{babel}

%T1 Zeichensatzkodierung
\usepackage[T1]{fontenc}

%Sonderzeichen der deutschen Sprache z.B. ß
\usepackage[ansinew]{inputenc}

%%%%%%%%%%%%%
%%Schriften%%
%%%%%%%%%%%%%

%Schrift ändern
%\renewcommand{\familydefault}{pag}

%Schriftarten ändern
\newcommand{\changefont}[3]{\fontfamily{#1} \fontseries{#2} \fontshape{#3} \selectfont}

% Latin Modern
%\usepackage{lmodern}
% -------------------
% Palantino , Helvetica, Courier
\usepackage{mathpazo}
\usepackage[scaled=.95]{helvet}
\usepackage{courier}

% Erlaubt automatische Trennung von Worten mit Umlauten
%\usepackage{ae}

%Zum definieren der einzelnen Elemente im Text. Sollte aber beim Standard belassen werden
\usepackage{typearea}

%Zum definieren der Zeilenabstandes
\usepackage{setspace}

%durch dieses Paket ist das Einbinden von Grafiken möglich, aber nur eps Dateien
\usepackage[]{graphicx}

%durch dies Pakete können Bilder mit Textumflossen werden
%\usepackage{floatflt}
%\usepackage{/picins}

%durch dieses Paket werden die Farbnamen definiert

\usepackage[usenames]{color}


%hierdurch wird Quellcode besser dargestellt
\usepackage{listings}
\renewcommand\lstlistlistingname{Quellcodeverzeichnis}
\renewcommand{\lstlistingname}{Quellcode}
\usepackage{xcolor}

%Abkürzungsverzeichnis
\usepackage[intoc]{nomencl}
\usepackage[normalem]{ulem} %Möglichkeiten zum welligen Unterstreichen bzw. durchstreichen von Text

\usepackage[acronym, toc,languages=german]{glossaries}
\newglossaryentry{signavio}{
	name=Signavio,
	description=ist eine Business-Process-Management-Software von SAP \cite[vgl.][]{sapExplorerSignavio}
}
\newglossaryentry{process-mining}{
	name=Process-Mining,
	description=ist eine wissenschaftliche Disziplin zwischen Data-Mining und Prozessmodellierung. Ziel ist das Entdecken\, die �berwachung und die Verbesserung echter Prozesse \cite[vgl.][S. 1]{westergaardProcessMiningManifesto}
}
\newglossaryentry{jira}{
	name=Jira,
	description=ist eine \glqq{}Software f�r die Vorgangs- und Projektverfolgung\grqq{} \cite{atlassianJiraSoftwareFur}
}
\newglossaryentry{celonis}{
	name=Celonis,
	description=ist eine \gls{process-mining}-Software vom gleichnamigen Unternehmen \cite[vgl.][]{celonisCelonisProcessAnalytics}
}
\newglossaryentry{confluence}{
	name=Confluence,
	description=ist eine Wiki-Software \cite[vgl.][]{altlassianConfluenceRemotefreundlicheArbeitsbereich}
}
\newglossaryentry{gitlab}{
	name=GitLab,
	description=ist eine Online-Dienst zur Versionsverwaltung von Softwareprojekten \cite[vgl.][]{gitlabDevSecOpsPlatform}
}
\newglossaryentry{profil}{
	name=profil,
	description=ist ein Gesch�ftsbereich der \gls{deg}\, welcher L�sungen zur F�rdermittelverwaltung vermarktet \cite[vgl.][]{dataexpertsgmbhProfilDataExperts}
}


\newacronym{deg}{deg}{data experts gmbh}
\newacronym{swot}{SWOT-Analyse}{St�rken, Schw�chen, Chancen, Risiken -- Analyse}
\newacronym{vrinos}{VRINOS-Analyse}{Value, Rarity, Imitablility, Non-Substitutability, Organization, Sustainability -- Analyse}
\newacronym{ag}{AG}{Arbeitsgemeinschaft}
\newacronym{iso}{ISO}{International Organization for Standardization}
\newacronym{bpmn}{BPMN}{Business Process Model and Notation}
\newacronym{api}{API}{Application Programming Interface}
\newacronym{csv}{CSV}{Comma-separated Values}
\newacronym{pg}{PG}{Projektgruppe}
\newacronym{qs}{QS}{Qualit�tssicherung}
\newacronym{dmn}{DMN}{Decision Model and Notation}
\newacronym{hkr}{HKR}{Haushalts-, Kassen- und Rechnungswesen}
\newacronym[longplural={geografischen Informationssystem}, shortplural={GIS}]{gis}{GIS}{geografisches Informationssystem}
\newacronym[longplural={dom�nenspezifischen Sprache}, shortplural={DSL}]{dsl}{DSL}{dom�nenspezifische Sprache}
\newacronym{rfk}{RFK}{Referenzfl�chenkataster}
\newacronym{gb}{GB}{Gesch�ftsbereich}
\newacronym{bmel}{BMEL}{Bundesministerium f�r Ern�hrung und Landwirtschaft}
\newacronym{gap}{GAP}{gemeinsame Agrarpolitik}
\newacronym{eu}{EU}{europ�ische Union}
\newacronym{ziaf}{ZIAF}{Zahlstelle InVeKoS Agrarf�rderung}
\newacronym{invekos}{InVeKoS}{integriertes Verwaltungs- und Kontrollsystem}
\newacronym{b2b}{B2B}{Business to Business}
\newacronym{b2c}{B2C}{Business to Customer}
\newacronym{ki}{KI}{k�nstliche Intelligenz}
\newacronym{crm}{CRM}{Customer Relationship Management}
\newacronym{xml}{XML}{Extended Markup Language}
\newacronym{wysiwyg}{WYSIWYG}{\glqq{}What you see is what you get\grqq{}}

\makeglossaries
%%%%%%%%%%%%%%%%%%%%
%%Allgemeine Dinge%%
%%%%%%%%%%%%%%%%%%%%

%Zeilenabstand 1.5
\onehalfspace 

%%%%%%%%%%%%%%%%%%%%%%%%%%%%%%%%%%%%%%%%%%
%%Definieren des Abkürzungsverzeichnises%%
%%%%%%%%%%%%%%%%%%%%%%%%%%%%%%%%%%%%%%%%%%

% Befehl umbenennen in abk
\let\abk\nomenclature

% Befehl damit in der Kopfzeile auf der zweiten Seite auch Abkürzungsverzeichnis angezeigt wird
\newcommand{\Abkuerzungsverzeichnis}{
%\clearpage % bei Option "oneside"
\cleardoublepage % bei Option "twoside"
\markboth{\nomname}{\nomname}
\printnomenclature
\newpage
}

% Deutsche Überschrift
\renewcommand{\nomname}{Abkürzungsverzeichnis}

% Punkte zw. Abkürzung und Erklärung
\setlength{\nomlabelwidth}{.20\hsize}
\renewcommand{\nomlabel}[1]{#1 \dotfill}

% Zeilenabstände verkleinern
\setlength{\nomitemsep}{-\parsep}
\makenomenclature

%Unterstreichung des Markup Buchstaben
\newcommand{\markup}[1]{\uline{#1}}

%Optionale Argument können mit dem voreingestellten Wert verglichen werden
\usepackage{ifthen}

%%%%%%%%%%%%%%%%%%%%%%%%%%%%%%%%%%%%%%
%%Definieren der Kopf- und Fußzeilen%%
%%%%%%%%%%%%%%%%%%%%%%%%%%%%%%%%%%%%%%

\usepackage{fancyhdr} 			%Paket laden
\pagestyle{fancy}						%eigener Seitenstil
\fancyhf{}									%alle Kopf- und Fußzeilenfelder bereinigen
\fancyhead[OR]{						 	% "O" steht für "odd", also ungerade Seiten; "L" steht für links
	\color{black}
	\changefont{pag}{m}{n}
	\bfseries\leftmark
}
\fancyhead[EL]{ % "E" für "even", also gerade Seiten; "R" steht für rechts
%\fancyhead[L]{
	\color{black}
	\changefont{pag}{m}{n}
	\bfseries\rightmark
}
\renewcommand{\chaptermark}[1]{\markboth{#1}{}}
\renewcommand{\sectionmark}[1]{\markright{\thesection\ #1}}
\renewcommand{\headrulewidth}{0.4pt}				%obere Trennlinie

\fancyfoot[OR,EL]{			%Seitennummer
%\fancyfoot[OR,L]{			%Seitennummer
	\color{black}
	\changefont{pag}{m}{n}
	\bfseries\thepage
}	
\renewcommand{\footrulewidth}{0.4pt}				%untere Trennlinie

%Plain umdefinieren für z.B. neues Kapitel
\fancypagestyle{plain}{
	\fancyhead{} 												
	\renewcommand{\headrulewidth}{0pt}
	\renewcommand{\footrulewidth}{0.4 pt}
}

%Farbe für Trennlinie im Kopf wechseln
\renewcommand{\headrule}{{\color{black}
	\hrule width\headwidth height\headrulewidth \vskip-\headrulewidth}
}

%Farbe für Trennlinie im Fuß wechseln
\renewcommand{\footrule}{{\color{black}
	\hrule width\headwidth height\headrulewidth \vskip-\headrulewidth \vspace{2mm}}
}

%zum definieren von Farben
\definecolor{myBlue} {rgb}{0.423529412,0.388235294,0.725490196}
\definecolor{myRed} {cmyk}{0.02,0.88,0.99,0.0}
\definecolor{myBlue2} {cmyk}{0.681,0.511,0,0.471}
\definecolor{myTablehead} {cmyk}{0.57,0.19,0,0}

%%%%
%%TABELLEN
%%%%

\usepackage{colortbl}
\usepackage{hhline,float}
\usepackage{array}
\usepackage{booktabs} %einzelne Linien in Tabellen dicker oder dünner
\usepackage{supertabular} %Tabellen über mehrere Seiten

%%%%
%%Tabellen vorbereiten
%%
%%Parameter
%%   Caption
%%   Label
%%   Table
%%%%
\newcommand{\ocTable}[4]{
	\begin{center}
		\begin{table}[htbp]
			#3
			\caption{#1}
			\label{tbl:#2}
			#4
		\end{table}
	\end{center}
}

%%%%
%%Tabellenkopf vorbereiten
%%%%
\newcommand{\ocTableHeadRow}{\rowcolor{myTablehead}}
\newcommand{\ocTablHead}[1]{
	\textcolor{white}{\textbf{#1}}
}
%%%%
%%Tabellensatzformatierungen von Herrn Voß
%%http://www.torsten-schuetze.de/tex/tabsatz-2004.pdf
%%%%
\newcolumntype{f}{>{$}l<{$}}
\newcolumntype{n}{l}
\newcolumntype{N}{>{\scriptsize}l}
\newcolumntype{v}[1]{>{\raggedright\hspace{0pt}}p{#1}}
\newcolumntype{V}[1]{>{\scriptsize\raggedright\hspace{0pt}}p{#1}}
\newcolumntype{B}[1]{>{\boldmath\DC@{.}{,}{#1}}l<{\DC@end}}
\newcolumntype{d}[1]{>{\DC@{.}{,}{#1}}l<{\DC@end}}
\newcolumntype{R}[1]{%
	>{\begin{turn}{90}\begin{minipage}{#1}\scriptsize\raggedright\hspace{0pt}}l%
	<{\end{minipage}\end{turn}}%
}
\newcolumntype{x}{>{\scriptsize\raggedright\hspace{0pt}}X}
\newcolumntype{A}{>{\columncolor{myTablehead}}c}

%%%%
%%Seitenverweise
%%%%
% wenn gleiche Seite dann keinen Seitenverweis
\newcommand{\ocRef}[2]{#1 \ref{#2}\label{label #2}\ifthenelse{\equal{\pageref{label #2}}{\pageref{#2}}}{}{, S. \pageref{#2}}}
\newcommand{\ocFigref}[1]{\ocRef{Abb.}{fig:#1}}
\newcommand{\ocTblref}[1]{\ocRef{Tabelle}{tbl:#1}}
\newcommand{\ocPararef}[1]{\ocRef{Abschnitt}{sec:#1}}
\newcommand{\ocChapterref}[1]{\ocRef{Kapitel}{sec:#1}}
\newcommand{\ocLstref}[1]{\ocRef{Quellcode}{#1}}
%\newcommand{\ocNameref}[1]{`\nameref{#1}' auf Seite~\pageref{#1}}

%To-Do Befehl
\newcommand{\todo}[1]{\textbf{\textsc{\textcolor{red}{(TODO: #1)\\}}}}
\newcommand{\vera}[1]{\textbf{\textsc{\textcolor{purple}{(Frage an Vera: #1)\\}}}}
\newcommand{\niels}[1]{\textbf{\textsc{\textcolor{blue}{(Frage an Niels: #1)\\}}}}
\newcommand{\plannedpages}[1]{\textbf{\textsc{\textcolor{green}{(Geplante Seitenzahl: #1)\\}}}}
\newcommand{\reviewng}[1]{\textbf{\textsc{\textcolor{orange}{(Anmerkung Niels: #1)}}}}
\newcommand{\reviewvm}[1]{\textbf{\textsc{\textcolor{orange}{(Anmerkung Vera: #1)}}}}

\newboolean{shownotes}
\setboolean{shownotes}{true}

\newcommand{\note}[1]{%
	\ifthenelse{\boolean{shownotes}}{%
		{\colorbox{gray!10}{%
			\parbox{1\linewidth}{	
					\linespread{1}\selectfont   % etwas mehr Zeilenabstand
					\setlength{\parskip}{6pt}
					\ttfamily 
					Leitfragen
					
					\footnotesize
					#1
				}%
			}%
		}%
	}{}%
}

\newcommand{\realnote}[1]{%
	\ifthenelse{\boolean{shownotes}}{%
		{\colorbox{gray!10}{%
				\parbox{1\linewidth}{	
					\linespread{1}\selectfont   % etwas mehr Zeilenabstand
					\setlength{\parskip}{6pt}
					\ttfamily 					
					\footnotesize
					#1
				}%
			}%
		}%
	}{}%
}

\newcommand{\beispiel}[1]{%
	\noindent
	\setlength{\fboxsep}{6pt} % Abstand zwischen Text und Rahmen
	\fcolorbox{black}{gray!10}{%
		\parbox{\dimexpr\linewidth-2\fboxsep-2\fboxrule}{%
			\textbf{Beispiel:} #1
		}%
	}%
}

% Benutzung:
%\beispiel{Dies ist ein Beispieltext ohne zusätzliche Pakete.}

\setlength\fboxsep{0.5pt}
\setlength\fboxrule{0.3pt}


\begin{document}
% deckblatt
\begin{titlepage}
\begin{center}

\vspace*{20mm}
{\large \bf KI-gest�tzte �bersetzung nat�rlichsprachlicher Pr�fungslogik in DSL-Skripte  \\[2mm]\small am Beispiel von Plausibilit�tspr�fungen f�r digitale Antragsformulare \bf }
\vspace*{12mm}

zur Erlangung des akademischen Grades\\
Master of Science\\
vorgelegt von\\
\large Florian Lemnitzer\\
\normalsize am \\
02.03.2025

%\includegraphics[scale=.8]{figs/deckblatt.png}\\
\small


\vspace*{12mm}
%\epsfig{figure=figs/logo-fhb.eps,height=100px}\\
\includegraphics[height=120px]{figs/2015_10_05_THB_FB-W_Logo_RGB.jpg}\\
\small 
Technische Hochschule Brandenburg\\
Fachbereich: Wirtschaft\\
Studiengang: Master Digitalisierung und Management\\
Laufendes Semester: 4. Semester\\
Modul: Angewandtes Change Management \\
E-Mail: lemnitze@th-brandenburg.de \\
Matrikelnummer: 20236222 \\
Erstgutachten: Prof. Dr. Vera Meister \\
Zweitgutachten: M. Sc. Niels Gundermann
%Betreuer B?

\end{center}
\end{titlepage}
\clearemptydoublepage
  
\pagenumbering{Roman}
\setcounter{page}{2}

\renewcommand{\baselinestretch}{0.8}
%\addtocontents{toc}{\protect\thispagestyle{empty}}
\tableofcontents
%\addcontentsline{toc}{chapter}{Inhaltsverzeichnis}
\renewcommand{\baselinestretch}{1.25}  

\clearemptydoublepage
\listoffigures
%\clearemptydoublepage
\listoftables
%\clearemptydoublepage


% nomenklatur
%\include{texs/002-nomenklatur}
%\Abkuerzungsverzeichnis
\printglossary[type=\acronymtype, title=Abk"urzungsverzeichnis, nonumberlist]
\printglossary[nonumberlist]
\clearemptydoublepage
\pagenumbering{arabic}
\setcounter{page}{1}
%---
\chapter{Einleitung}
\plannedpages{3 - 6 | T: 01.02.26}
\note{
Was ist das konkrete Thema in einem Satz?

Warum ist das Thema relevant f�r Praxis und Forschung?

Welches reale Problem adressiert die Arbeit (konkret: Plausibilit�tspr�fungen in digitalen Formularen)?

F�r welche Zielgruppe ist die Arbeit relevant (Entwickler, Fachanwender, Beh�rden, Forschende)?

Welchen Beitrag liefert die Arbeit knapp zusammengefasst?

Wie ist die Struktur der Arbeit (kurzer Fahrplan der Kapitel)?

Welche Begriffe/Abk�rzungen werden zentral verwendet?
}
\clearemptydoublepage
\chapter{Problemstellung}
\plannedpages{3 - 6 | T: 16.11.25}
\clearemptydoublepage
\chapter{Ziele und Hypothese}
\plannedpages{3 - 6 | T: 23.11.25}
\note{
Was sind die operativen Ziele der Arbeit (z. B. Prototyp, Evaluationskriterien, Metriken)?

Welche forschungsleitende Hypothese(n) werden getestet?

Welche messbaren Erfolgskriterien definierst du (z. B. Genauigkeit, Zeitersparnis, Verst�ndlichkeit, Fehlerreduktion)?

Welche Nebenbedingungen gelten (z. B. verwendete DSL, Datengrundlage, Datenschutz)?

Welche Annahmen werden explizit getroffen und wie werden sie �berpr�ft?
}
\clearemptydoublepage
\chapter{Forschungsstand}
\plannedpages{6 - 9 | T: 07.12.25}
\clearemptydoublepage
\chapter{Methode}
\plannedpages{9 - 12 | T: 04.01.26}
\note{
Welchen methodischen Ansatz w�hlst du (gestaltungsorientiert: Design Science, Prototyping, Mixed-Methods, Usability-Testing)?

Welche Artefakte wirst du entwerfen (DSL-Erweiterung, Konversionspipeline, Nutzerinterface, Evaluationsskripte)?

Welche Datengrundlage verwendest du (Anwendungsf�lle, Formularbeispiele, Annotatoren, synthetische Daten)?

Wie sieht der Entwicklungsworkflow aus (Iterationen, Validationsschritte, Tools, Versionierung)?

Welche KI-Modelle und Prompt-/Fine-tuning-Strategien werden eingesetzt?

Welche Metriken und Messmethoden nutzt du zur Bewertung (Precision/Recall, BLEU/ROUGE ungeeignet? � nenne passendere Metriken: Genauigkeit der Regel�bersetzung, menschliche Lesbarkeit, Zeitaufwand)?

Welche Nutzerstudien oder Experten-Reviews planst du (Anzahl, Auswahlkriterien, Aufgaben, Messinstrumente)?

Wie stellst du Reproduzierbarkeit und Validit�t sicher (Artefakte, Skripte, Protokolle, Datenschutz)?

Welche ethischen und rechtlichen Fragen beachtest du (Bias, Datenherkunft, Verarbeitung personenbezogener Daten)?
}

\section{Plausibilit�tspr�fungen}
F�r eine Live-Validierung (auch Plausibilit�tspr�fung genannt) werden im Formularverwalter folgende Eigenschaften definiert: Name, Fehlertext, Beschreibung, Fehlerlevel und verkn�pfte Formulare und Antragsverfahren.

\subsubsection{Name}
Der Name ist der Bezeichner der Vorgabe. Er ist nicht eindeutig -- es kann mehrere Vorgaben mit dem gleichen Namen geben. Die Auftraggeber sind dazu aufgefordert die Namen technisch auswertbar zu gestalten, indem sie auf die Verwendung von Umlauten, Sonderzeichen und Leerzeichen im Namen verzichten. Das wird nicht erzwungen und deshalb nicht bei allen Vorgaben ber�cksichtigt. 

Beispiel: \emph{IBAN\_Laenge}

\subsubsection{Fehlertext}
In diesem Feld wird der Text eingetragen, der dem Anwender angezeigt werden soll. Meistens wird genau der Text dem Anwender angezeigt. Manchmal werden Platzhalter verwendet. Wie diese zu ersetzen sind, geht meistens aus der Beschreibung hervor.

Beispiel: \emph{Die von Ihnen eingegebene IBAN hat nicht die erforderlichen 22 Stellen.}

\subsubsection{Beschreibung}
Die Beschreibung enth�lt die Logik, die bestimmt, unter welchen Bedingungen die Meldung ausgegeben werden soll. Die Logik wird in nat�rlicher Sprache beschrieben. Die Qualit�t der Beschreibungen variiert stark -- abh�ngig von der Komplexit�t der Pr�fung und dem Hintergrund des Autors.

In der Beschreibung k�nnen Felder aus den Verkn�pften Formularen referenziert werden. Wenn das gemacht wird, kann die Referenz technisch zu einem Feld im Formular zugeordnet werden. In manchen F�llen wird nur der technische Bezeichner eines Formularfeldes (wie \emph{bi\_iban} im unten stehenden Beispiel) oder nur eine nat�rlichsprachliche Umschreibung (bspw. \glqq{}In der ersten Spalte der Tabelle\grqq{}) verwendet.

Beispiel \emph{bi\_iban: Wenn es sich um eine DE - IBAN handelt, muss sie incl. DE genau 22 Zeichen lang sein.}

\subsubsection{Fehlerlevel}
Das Fehlerlevel ist der Schweregrad der Pr�fung. Es reicht von Hinweis bis Disqualifikation. Es wird �ber ein Auswahlfeld im Formularverwalter erfasst. Es ist dadurch eindeutig und technisch auswertbar. Es gibt 5 Level:
\begin{description}
	\item[Information, Warnung, Fehler] Diese Level sind rein informativ. Die Meldungen werden angezeigt, schr�nken die Bearbeitung aber nicht weiter ein.
	\item[Fataler Fehler] Enth�lt der Antrag ein Formular mit einem fatalen Fehler, kann er nicht eingereicht werden.
	\item[Disqualifikation] Ein Disqualifikationsfehler schlie�t ein Formular vom Antrag aus. Der Antrag kann eingereicht werden, aber das entsprechende Formular wird nicht mit eingereicht.
\end{description}

Beispiel: \emph{Fataler Fehler}

\subsubsection{verkn�pfte Formulare und Antragsverfahren}
Eine Plausi-Vorgabe ist mit den Antragsverfahren verkn�pft, f�r die sie gelten soll. Und sie ist mit den Formularen Verkn�pft, auf die f�r die Pr�fung zugegriffen werden muss. 

Beispiel: \\
\emph{Antragsverfahren: Brandenburg Agrar-Antrag 2026, Brandenburg ELER-Antrag 2026, Mecklemburg-Vorpommern Agrar-Antrag 2026, Mecklemburg-Vorpommern ELER-Antrag 2026, Schleswig Holstein Agrar-Antrag 2026}\\
\emph{Dokument: Stammdaten}

\section{Implementierung}
Diese Vorgaben werden von der data experts mit Hilfe der propriet�ren Skriptsprache \glqq{}formScript\grqq{} implementiert. Diese ist f�r die Verarbeitung und Manipulation von Formulardaten geschaffen worden. Die Skripte werden als \gls{xml}-Dateien im Quellcode hinterlegt. Zur Compile-Zeit werden sie zu Java-Code �bersetzt. \ocLstref{iban-laenge} ist das Skript zu der Vorgabe \glqq{}IBAN\_Laenge\grqq{}, die oben als Beispiel verwendet wurde.

\begin{lstlisting}[language=xml, caption={IBAN\_Laenge Skript}, label=iban-laenge]
	<?xml version="1.0" encoding="UTF-8"?><scripts>
	<script name="MC26_STDA_1_IBAN_Laenge"><![CDATA[
	@meta name: "MC26_STDA_1_IBAN_Laenge";
	
	master stda1 as #202617101;
	
	var iban = stda1->bi_IBAN@value;
	if(Strings.startsWith(iban, "DE") 
	&& Strings.length(iban) != 22) {
		fatal [stda1->bi_IBAN]
		: "Die von Ihnen eingegebene IBAN hat nicht die 
		erforderlichen 22 Stellen."
		: meta {errorId: "IBAN_Laenge"};
	}]]></script>
	</scripts>
\end{lstlisting}
\clearemptydoublepage
\chapter{Ergebnisse}
\plannedpages{9 - 15 | T: 18.01.26}
\note{
Welche Artefakte sind entstanden (DSL-Spezifikation, Konverter, Beispielskripte, Evaluationsdatensatz)?

Welche quantitativen Ergebnisse liegen vor (Metriken, Tabellen, statistische Tests)?

Welche qualitativen Ergebnisse liegen vor (Expert:innen-Feedback, Fallstudien, Fehlerklassen)?

Wie verhalten sich die Ergebnisse gegen�ber den definierten Erfolgskriterien?

Welche typische Fehlerf�lle/Fehlermodi traten auf? Gib konkrete Beispiele.

Welche Designentscheidungen haben sich als besonders wirkungsvoll erwiesen?

Welche Artefakte oder Prototyp-Versionen sind reproduzierbar und wo liegen Einschr�nkungen?
}
\clearemptydoublepage
\chapter{Diskussion}
\plannedpages{9 - 12 | T: 25.01.26}
\note{
Was bedeuten die Ergebnisse f�r die Hypothese? Best�tigt, modifiziert oder widerlegt sie die Hypothese?

Welche Ursachen haben beobachtete Fehler oder Abweichungen? Technisch und methodisch getrennt.

Wie verallgemeinerbar sind die Ergebnisse (Kontexte, Domains, Formulartypen)?

Welche Implikationen haben die Ergebnisse f�r Praxis (Implementierungsempfehlungen) und Forschung (neue Fragen)?

Welche Limitationen hatte die Studie (Datenmenge, Expert:innenzahl, Modellgrenzen, Metrik-Validit�t)?

Wie w�rden alternative Methoden oder andere DSL-Designs das Ergebnis wahrscheinlich ver�ndern?

Welche Risiken und Nebenwirkungen (z. B. Fehlinterpretation durch Fachanwender, Maintenance-Aufwand) sind relevant?
}
\clearemptydoublepage
\chapter{Fazit}
\plannedpages{3 - 6 | T: 01.02.26}
\note{
Was sind die knappsten, pr�fungsrelevanten Erkenntnisse der Arbeit?

Welche konkreten Empfehlungen gibst du f�r Entwickler, Beh�rden oder Forscher?

Welche offene Forschungsfragen bleiben und welche n�chsten Schritte empfiehlst du (konkrete Experimente, Skalierung, Integration)?

Welche praktische Bedeutung hat die Arbeit kurz- und mittelfristig?

Welche Learnings f�r dein eigenes Vorgehen sind wichtig (Was w�rdest du beim n�chsten Mal anders machen)?
}
%---

% --- ! Erzeugen des Abbildungsverzeichnisses ! ---
%\addcontentsline{toc}{chapter}{\numberline{II}\space Abbildungsverzeichnis}

%\pagenumbering{Roman}
%\setcounter{page}{3}

%\part{Verzeichnisse und Anhang}



%\listoftables

% --- ! Erzeugen des Literaturverzeichnisses ! ---!
%\addcontentsline{toc}{chapter}{\numberline{III}\space Literaturverzeichnis}
\bibliographystyle{plainnat}
%\bibliographystyle{harvard}
\bibliography{./../../MyLibrary}
\clearemptydoublepage

% anhaenge
%\begin{appendix}
%\label{anhang}
 %\chapter{Allgemeine Erg"anzungen}

\section{ChatGPT}
\label{chatgpt}
Im folgenden ist eine Interaktion mit GPT-3.5 protokolliert. Auf die \gls{ki} wurde �ber die Webseite \cite{openaiChatGPT} zugegriffen. Das Vorgehen und die Leitfragen aus dem Prompt sind aus \cite{tiffertGrundlagenOperativenVertriebsmanagement2020} entnommen.
\subsubsection{Prompt}
Ich habe folgenden Prozess zur Teilnahme an einer Ausschreibung f�r einen Auftrag eines Ministeriums:\\
Der Prozess zur Neukundengewinnung beziehungsweise der Teilnahme an einer Ausschreibung k�nnte mit der Suche nach geeigneten Ausschreibungen beginnen. Darauf w�rde eine Evaluierung der Ausschreibungen auf Machbarkeit folgen. Danach k�men gegebenenfalls Angebotserstellung und -abgabe und im Falle einer Zusage eine Vertragserstellung und -unterzeichung.

Bitte definiere mir Kompetenzen, die zur Umsetzung der konkreten Aufgaben in dem Prozess notwendig sind. Nutze daf�r folgendes Vorgehen:\\
1. Reflexion der Schl�sselaufgaben im Rahmen der Verkaufsprozesse, 2. Reflexion der erfolgskritischen Verhaltensweisen, 3. Ableitung von Kompetenzen, 4. Formulierung von definierenden Verhaltensbeschreibungen.

Nutze f�r 2. die Leitfrage: Was muss man machen, um die Schl�sselaufgaben erfolgreich zu bew�ltigen?\\
Nutze f�r 3. die Leitfragen: Welche der obigen Beschreibungen zielen im Kern auf eine gleiche Grundverhaltensweise? Wie kann die jeweilige Gruppe mit einem zentralen Begriff bezeichnet werden?
\subsubsection{Antwort}
Hier ist eine m�gliche Herangehensweise:

Reflexion der Schl�sselaufgaben im Rahmen der Verkaufsprozesse:\\
a. Suche nach geeigneten Ausschreibungen\\
b. Evaluierung der Ausschreibungen auf Machbarkeit\\
c. Angebotserstellung und -abgabe\\
d. Vertragserstellung und -unterzeichnung\\

Reflexion der erfolgskritischen Verhaltensweisen:

Was muss man machen, um die Schl�sselaufgaben erfolgreich zu bew�ltigen?\\
a. Effektive Recherche- und Analysef�higkeiten, um geeignete Ausschreibungen zu identifizieren und zu verstehen.\\
b. Bewertungskompetenz, um die Machbarkeit der Ausschreibungen zu beurteilen und Potenziale sowie Risiken zu erkennen.\\
c. �berzeugende Kommunikationsf�higkeiten f�r die Erstellung und Pr�sentation von Angeboten.\\
d. Verhandlungsgeschick und rechtliches Verst�ndnis f�r die Vertragsverhandlungen und -abschl�sse.\\
Ableitung von Kompetenzen:

Welche der obigen Beschreibungen zielen im Kern auf eine gleiche Grundverhaltensweise?\\
a. Recherche- und Analysef�higkeiten sowie Bewertungskompetenz k�nnen als \glqq{}Informationsverarbeitung\grqq{} zusammengefasst werden.\\
b. Kommunikationsf�higkeiten und Verhandlungsgeschick k�nnten als \glqq{}Interaktion und �berzeugung\grqq{} betrachtet werden.\\
Formulierung von definierenden Verhaltensbeschreibungen:\\
a. Informationsverarbeitung: F�higkeit zur effizienten Recherche und Analyse von Ausschreibungen, um relevante Informationen zu extrahieren und zu bewerten.\\
b. Interaktion und �berzeugung: F�higkeit zur klaren und �berzeugenden Kommunikation sowie zum erfolgreichen Verhandeln w�hrend des Angebots- und Vertragsprozesses.\\

Diese beiden Gruppen von Kompetenzen - Informationsverarbeitung und Interaktion/�berzeugung - sind entscheidend f�r den erfolgreichen Abschluss des Ausschreibungsprozesses und k�nnen als grundlegende Anforderungen f�r die Teilnahme an Ausschreibungen betrachtet werden.

 %\clearemptydoublepage
 %\end{appendix}

% erklaerung
\addcontentsline{toc}{chapter}{Ehrenw"ortliche Erkl"arung}
\thispagestyle{empty}

\large
\begin{flushright}
  Brandenburg, den 31. August 2025
\end{flushright}

\vspace*{50mm}

Hiermit versichere ich, {\scshape Florian Lemnitzer}, dass ich die vorliegende Arbeit selbstst�ndig verfasst
und keine anderen als die angegebenen Quellen oder Hilfsmittel benutzt habe
und dass die Arbeit in gleicher Form noch keiner anderen
Pr�fungsbeh�rde vorgelegt wurde.


\vspace*{50mm}

\begin{flushright}
	\includegraphics[width=0.4\textwidth]{./figs/signature.png}\\
  {\scshape Florian Lemnitzer}
\end{flushright}

\normalsize


\end{document}